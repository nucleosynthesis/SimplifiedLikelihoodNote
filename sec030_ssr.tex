%%____________________________________________________________________________||
\section{Aggregate signal regions}
\label{sec:aggregate-signal-regions}

This section describes how the categorisation used by searches for new
physics may be simplified without breaking the correlation model. 
This may be done with the original likelihood or, approximately,
with the predictions and covariance matrix.

To define an aggregate region, $I$, based on the likelihood described in equation
\ref{eq:poisson-likelihood}, the probability to observe $n_{I}$ events is given by

\begin{equation}
 P_{I}(\mu) = \dfrac{(\mu \cdot \sum_i(s_{i}+b_{i} \cdot \rho_{i}))^{n_{I}} e^{-\sum_i(\mu \cdot s_{i}+b_{i}\cdot\rho_{i})} }{n_{I}!}
\label{eq:agg-likelihood}
\end{equation}

where the sum is over all regions being aggregated. The aggregate regions
can then be used to define predictions and covariance that may be
used for the aggregate regions.

An alternative derivaration of the predictions and covariance of aggregate region 
required for the simplified likelihood definition in \ref{eq:full-likelihood} can
be made using the predictions and covariance of the nominal signal regions.
These can be merged using \ref{eq:agg-cov}. For the covariance of the aggregate
region to be accurate relies on the same conditions on the nuisances outlined 
in Section~\ref{sec:simplified-likelihood}. 

\begin{align}
b_{I} = \sum_i b_{i} && V_{IJ}=\sum_{ij}V_{ij}
\label{eq:agg-cov}
\end{align}

where the sum is over all regions being aggregated. The aggregate predictions
and covariance can then be used to define the simplified likelihood.

%%____________________________________________________________________________||
\section{Introduction}
\label{sec:intro}

Searches for new physics beyond the standard model (BSM) by the CMS collaboration are performed using a wide variety of 
strategies and using events with different final states and kinematic properties.  Often, 
the results of these searches are presented in terms of ``model-independent'' limits on the production 
cross-section for some new BSM particle. Common examples are searches for resonances whose decay products 
can be experimentally reconstructed with a high resolution resulting in narrow invariant mass peaks which can be readily distinguished from 
a smoothly varying background. Several searches, however, involve the use of final states with poor resolution or involving 
quantities such as the missing transverse momentum or the angles between objects in the final state. These searches are typically performed using the 
distributions of these quantities for which the separation between standard model (SM) processes and BSM signals is limited. Furthermore, 
sensitivity to a wide range of BSM physics can often be improved using a categorisation of events based on the number of a particular 
object in the event, such as charged leptons or jets, or based on a multivariate analysis (MVA) of the kinematics and/or reconstruction and 
identification quality of the final state particles in the event. For such searches, limits can only be expressed in terms of the 
parameter space of some specific complete or simplified BSM model.

Searches for BSM physics are often interpreted in only a small subset of the new physics
models for which they may be sensitive and must be re-interpreted by those
outside the CMS collaboration to evaluate their impact on generic models of new physics.
For example, the mastercode collaboration re-interprets searches to provide constraints
on GUT scale models of supersymmetry (SUSY)~\cite{mastercode}. For the reinterpretation the signal contribution 
is typically determined using an event generator such as {\sc Pythia}~\cite{pythia} followed by a simulation of the detector 
response and resolution using tools such as {\sc Delphes}~\cite{delphes} or by matching truth-level particles to
the reconstructed objects according to the re-ported detector performances. 
The background model and associated systematic uncertainties for the reinterpretation often rely
on simplfying assumptions which can cause bias in the result by, for example, breaking the 
correlation model used in the search. 

In this note, a procedure for reliably re-interpreting BSM physics searches using a reduced set of information
 on the background model and systematic uncertainties, which can be provided in CMS publications, is presented.



%%____________________________________________________________________________||

%%____________________________________________________________________________||
\section{Introduction}
\label{sec:intro}

Searches for new physics beyond the standard model (BSM) by the CMS collaboration are performed using a wide variety of 
strategies and using events with different final states and kinematic properties.  Often, 
the results of these searches are presented in terms of ``model-independent'' limits on the production 
cross-section for some new BSM particle. Common examples are searches for resonances whose decay products 
can be experimentally reconstructed with a high resolution resulting in narrow invariant mass peaks which can be readily distinguished from 
a smoothly varying background~\cite{Khachatryan:2016yec}. Several searches, however, involve the use of final states with low mass resolution or involving 
quantities such as the missing transverse momentum (as in~\cite{Khachatryan:2011tk,Khachatryan:2016mdm}) or the angles between objects in the final state (as in~\cite{Khachatryan:2015pua}). 
These searches are typically performed using the 
distributions of these quantities for which the separation between standard model (SM) processes and BSM signals is limited. Furthermore, 
sensitivity to a wide range of BSM physics can often be improved using a categorisation of events based on the number of a particular 
object in the event, such as charged leptons or jets, or based on a multivariate analysis (MVA) of the kinematics and/or reconstruction and 
identification quality of the final state particles in the event. For such searches, limits can only be expressed in terms of the 
parameter space of some specific complete or simplified BSM model.

Searches for BSM physics are often interpreted using a small subset of new physics 
models serving as benchmarks for the sensitivity of the search. Often, the searches are re-interpreted 
to provide constraints on other models of new physics, not included in the publication.
Re-interpretations can also be provided using complete models of new physics and the constraints from 
these searches are combined with measurements and searches from other experiments~\cite{mastercode}. 
For these reinterpretations, the signal contribution is typically determined using an event generator 
such as {\sc Pythia}~\cite{pythia} followed by a simulation of the detector 
response and resolution using tools such as {\sc Delphes}~\cite{delphes} or by matching generated particles to
the reconstructed objects according to published information regarding the performance of the CMS detector. 
The background contributions to search regions and the associated systematic uncertainties, however, often rely
on simplifying assumptions, in particular where the search is performed using multiple event categories or 
the distributions of one or more discriminating variables, which can lead to inaccuracies in the re-interpretation. 

In previous publications, the total background predictions  
and systematic uncertainty has been provided by searches in each region for which the contribution 
from potential BSM signals are expected to be significant. 
Typically these searches also define 'super' regions which cover larger regions of the relevant discriminating variables 
than those used in the analysis. This allows re-interpretation of the results by selecting the most sensitive 
super region for a given BSM interpretation. This procedure avoids the need to provide correlations between the 
distributions of the discriminating variables. While the procedure is robust, the loss of information included in the regions 
neglected for each BSM scenario can can result in a significant loss of sensitivity. 

In this note, an alternative procedure for re-interpreting BSM physics searches by approximating
the full background model and systematic uncertainties is presented. The procedure uses a reduced 
set of information to describe the background model 
and the correlations between different regions used in the searches, minimising the loss of sensitivity.



%%____________________________________________________________________________||

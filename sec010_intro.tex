%%____________________________________________________________________________||
\section{Introduction}
\label{sec:intro}
In order to achieve sensitivity to a wide range of new physics models searches
for BSM physics typically use fine categorisations of events in the signal region.
For example, several searches for Supersymmetry (SUSY) using data from Run 2 of the LHS
used $\mathcal{O}100$ of bins \cite{susy-searches}. These searches are interpreted 
using the simplified models described in \ref{sec:intro}, providing an indication
of the topologies to which these searches are sensitive. These interpretations can 
be used by theorists to estimate the impact of the search on more complicated models 
\cite{usingSimpForModels}, however, 

\begin{itemize}
\item What is recasting? Overview of methods?
\item Problems in recasting complicated analyses
\item Possible ways to allow recasters to improve accuracy (SSR + SLH)
\item Relevant details of AlphaT search
\end{itemize}


%%____________________________________________________________________________||

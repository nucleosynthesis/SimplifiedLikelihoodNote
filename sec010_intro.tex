%%____________________________________________________________________________||
\section{Introduction}
\label{sec:intro}

Searches for new physics beyond the standard model (BSM) are performed using a wide variety of 
strategies and using events with different final states and kinematic properties.  Often, 
the results of these searches are presented in terms of ``model-independent'' limits on the production 
cross-section for some new BSM particle. Common examples are searches for resonances whose decay products 
can be experimentally reconstructed with a high resolution resulting in narrow invariant mass peaks which can be readily distinguished from 
a smoothly varying backround. Several searches, however, involve the use of final states with poor resolution or involving 
quantities such as the missing transverse momentum \ptvecmiss or angular variables. These searches are typically performed using the 
distributions of these quantities for which the separation between standard model (SM) processes and BSM signals is limited. Furthermore, 
sensitivity to a wide range of BSM physics can often be improved using a categorisation of events based on the number of a particular 
object in the event, such as charged leptons or jets, or based on a multivariate analysis (MVA) of the kinematics and/or reconstruction and 
identification quality of the final state particles in the event. For such searches, limits can only be expressed in terms of the 
parameter space of some specific complete or simplified BSM model.

While re-interpretations of these types of searches are possible, they must often rely heavily on a simplified simulation of 
the detector response and resolution in order to model the expected contributions from SM processes and their uncertainties. 
In this note, a procedure for re-interpreting BSM physics searches using a reduced set of information on the model used for the 
expected background and signal processes contributing to the search is presented. 

Sec 1 introduction to terminology 
Sec 2 Simplified Likelihood 
Sec 3 validation exercises



%%____________________________________________________________________________||

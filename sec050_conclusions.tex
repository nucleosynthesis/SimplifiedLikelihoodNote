%%____________________________________________________________________________||
\section{Summary}
\label{sec:conclusions}
Searches for new physics by the CMS collaboration are performed using a wide variety of 
strategies and using events with different final states and kinematic properties. 
Re-interpreting these searches requires approximating the background model
and associated systematic uncertainties for the search. This can be 
achieved by defining a simplified likelihood combining one or more search regions in which the systematic 
uncertainties on the backgrounds are modelled  as Gaussian  constrained nuisance parameters. 
To define the simplified likelihood requires only the observed data, total expected background contributions and 
the covariance matrix describing the correlations of the total background between each of the search regions. 
This information can be provided readily in CMS publications. In addition, the number of search regions being 
considered can be reduced, where necessary, through the use of aggregated search regions.  
The use of the simplified likelihood allows a consistent and well defined approximation of the 
full likelihood used by CMS searches, allowing for re-interpretation of the search under different BSM 
physics models.

%%____________________________________________________________________________||

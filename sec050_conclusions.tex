%%____________________________________________________________________________||
\section{Conclusions}
\label{sec:conclusions}
Searches for new physics by the CMS collaboration are performed using a wide variety of 
strategies and using events with different final states and kinematic properties. 
Re-interpreting these searches requires approximating the background model
and associated systematic uncertainties for the search. This can be 
achieved by definning a binned simplified likelihood where the systematic 
uncertainties on the backgrounds are approximated as correlated guassian constraints between the bins. 
To define the simplfied likelihood requires only the background predictions and covariance matrix which may be 
included in CMS publications. In addition, the number of search regions that must
be considered can be reduced, where necessary, through the use of aggregate regions. In summary, 
the use of the simplified likelihood allows a consistent and reliable approximation of the 
full likelihood used by CMS searches.



%%____________________________________________________________________________||

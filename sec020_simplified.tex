%%____________________________________________________________________________||
\section{Simplified likelihood}
\label{sec:simplified-likelihood}

While several methods for extracting 
limits are used, the most common approach follows the frequentist paradigm using the procedure 
described in Refs.~\cite{} and ~\cite{}. 
The observed data are interpreted using a likelihood defined as,

\begin{equation}
 \mathcal{L}(\mu\cdot s(\boldsymbol{\theta}) + b(\boldsymbol{\theta})) = 
 \mathcal{P}(\mathrm{data}|\mu\cdot s(\boldsymbol{\theta}) + b(\boldsymbol{\theta})) \cdot p(\tilde{\boldsymbol{\theta}}|\boldsymbol{\theta})
\label{eq:generic-likelihood}
\end{equation}

where $\mathcal{P}(\mathrm{data}|\mu\cdot s(\boldsymbol{\theta}) + b(\boldsymbol{\theta}))$ is a product of the proability 
over all events or bins in one or more discriminating variables or event categories to observe the data. The parameters 
$\boldsymbol{\theta}=\left(\theta_{1},\theta_{2}...\right)$ are nuisance parameters, which are used to model the variation of the 
signal $s(\boldsymbol{\theta})$ and background $b(\boldsymbol{\theta})$ models due to systematic uncertainties. Often, these nuisance 
parameters are constrained by external measurements, $\hat{\boldsymbol{\theta}}$, which are encoded in the 
probability density function $p(\tilde{\boldsymbol{\theta}}|\boldsymbol{\theta})$. 
The parameter $\mu$, typically referred to as a signal strenth, is a common scale factor for the expected signal contribution. 
A particular BSM model predicting $s(\boldsymbol{\theta})$ is said to be excluded at some confidence level 
when every value of $\mu\ge1$ is excluded at at least that confidence level.

The profiled likelihood ratio is defined as,

\begin{equation}
LLR
\label{eq:generic-likelihood}
\end{equation}


In order to acheive a good senstivity to a wide range of BSM models, searches are often performed 
by categorising events with different final states or according to some discriminating variable. 
The precise modelling of the backgrounds can therefore be rather complicated, involving hundreds 
of nuisance parameters. For this reason, the full likelihood is therefore often 
too detailed to describe in a CMS publication and includes too many (nuisance) parameters to provide numerically.

The following describes a procedure for using a reduced set of information provided by the CMS Collaboration to 
re-interpret searches for new physics through the use of a simplified likelihood. 
In practise, the likelihood defined in Equation~\ref{eq:generic-likelihood} can include regions in which negligible signal is expected under a wide 
range of BSM models. Typically these are referred to as ``control'' regions in that they allow to constrain the nuisance parameters that  
cause large variations in the background model.
In this note, the distinction is made between these regions and ``search'' regions, which instead 
are expected to include contrubutions from the signal, under some particular set of BSM models. A search region 
is defined by a set of criteria used to select events. These criteria can include categorisations based on 
the number of a certain type of object in the event such as jets or 
charged leptons, and intervals in some discriminating variable such as \ptvecmiss. The data in each search region, $i$, is characterised 
by a single number, $n_{i}$ which is the observed number of events. The likelihood is therefore constructed from a product of counting 
experiments, representing each search region in one or more search regions. 
For a given search region, $i$, the probability to observe $n_{i}$ events is given by

\begin{equation}
 P_{i}(\mu) := P(d_{i}|\mu \cdot s_{i}+b_{i}) = \dfrac{(\mu \cdot s_{i}+b_{i})^{n_{i}} e^{-(\mu \cdot s_{i}+b_{i})} }{n_{i}!}
\label{eq:poisson-likelihood}
\end{equation}

where $s_{i}$ and $b_{i}$ are the total expected signal and background contributions.
\footnote{In the case that the search region $R_{i}$ is a bin or interval in a distribution of some observable $x$, for which the signal and background models $s(x)$ 
and $b(x)$ are continuous functions of $x$ the values of $s_{i}$ and $b_{i}$ are taken as 
$s_{i}=\int_{R_{i}} s(x)dx$ and $b_{i}=\int_{R_{i}} b(x)dx$.} 
The likelihood for a search containing $N$ search regions is constructed as the product 
of the probabilities across the $N$ search regions, 

\begin{equation}
\mathcal{L}(\mu) = \prod_{i=1}^{N} P_{i}(\mu)
\label{eq:stat-likelihood}
\end{equation}

In most cases, the background contribution in each search region will not be known with perfect accuracy and is therefore 
subject to systematic uncertainties. These uncertainties are modelled by modifying the background contributions as 
$b_{i}\rightarrow b_{i}+\delta b_{i}$, where $\delta\mathbf{b}=(\delta b_{1},\delta b_{2}...\delta b_{N})$ are constrained nuisance parameters. The likelihood then takes the form

\begin{equation}
\mathcal{L}(\mu, \delta \mathbf{b}) = \prod_{i=1}^{N} P_{i}(\mu,\delta b_{i}) \cdot \mathrm{exp} \left[ \sum_{j=1}^{N}\sum_{k=1}^{N} (\delta b_{j}) V_{jk} (\delta b_{k}) \right]
\label{eq:full-likelihood}
\end{equation}

where $P_{i}(\mu,0)=P_{i}(\mu)$ and $V_{jk}$ represents the covariance between
the total expected background in the $j$--th and $k$--th search regions.
From Equation~\ref{eq:full-likelihood}, there are three ingredients which are 
required in order to construct the simplified likelihood. These ingredients are as follows; 

\begin{itemize}
\item {The number of observed events in each of the search regions, $n_{i}$.}
\item {The background and signal expectations in each search region, $b_{i}$ and $s_{i}$. The former is often derived using data driven methods and 
tuned simulation while the latter can be obtained from a number of publically available event generators and tools that simulate the response and resolution of the CMS detector.}
\item {The covariance between search regions for the backgrounds. These are either derived by simple error propogation of the background systematics to the expected background yields or 
using pseudo-experiments.}
\end{itemize}
 
It should be noted that the simplified likelihood presented in Equation~\ref{eq:full-likelihood} is an approximation to the full likelihood used 
by most CMS analyses in that it requires the following assumptions;

\begin{itemize}
\item{The constraints on the background contributions are Gaussian such that the distribution of the number of background events is symmetric about the expectation, $b_{i}$, 
and its variance is independent of $\delta \mathbf{b}$. Often, the background contributions are estimated from control regions in data with large sample sizes, which allows for this 
assumption to be made.}

\item{The linear correlation between the background contribution in each region is sufficient to model the  pdf of the background expectations such that the constraint 
can be expressed as a multivariate Gaussian.}
\end{itemize}

% covariance must be taken as the result of a maximum likelihood fit considering the control regions only. 
% If the control region is not included then the results fit may be taken. 
%%____________________________________________________________________________||
% \subsection{Motivation}
% \begin{itemize}
% \item Even with SSR recasters have insufficient information to reproduce analysis
% \item Full likelihood is overkill - what is necessary?
% \end{itemize}
% \subsection{Theory}
% \label{sec:sl-theory}
% \begin{itemize}
% \item Definition of simplified likelihood
% \item Inputs needed
% \item What's simplified? No signal systs, no CR, gaussian unc
% \end{itemize}
% \subsection{Procedure}
% \label{sec:sl-procedure}
% \begin{itemize}
% \item Determining correlation matrix
% \item Defining likelihood
% \item For study of impact setting off-diag elements to 0
% \end{itemize}
% \subsection{Signal contamination}
% \label{sec:signal-contamination}
% \begin{itemize}
% \item Definition of reduced efficiency method
% \item What do recasters need to take account of contamination?
% \end{itemize}
% \subsection{Results}
% \begin{itemize}
% \item Covariance matrix
% \item DeltaNLL vs r for example model (compare with + without correlations)
% \item Limit planes for several models (T1qqqq, T2tt, T2bb)
% \item Ratios + comparison to full
% \end{itemize}

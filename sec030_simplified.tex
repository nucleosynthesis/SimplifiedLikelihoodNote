%%____________________________________________________________________________||
\section{Simplified likelihood}
\label{sec:simplified-likelihood}

This section describes a procedure for using the information provided by the CMS Collaboration to 
re-interpret searches for new physics through the use of a simplified likelihood. The likelihood is 
constructed from a product of counting experiments, representing each search bin in one or more search regions. 
For a given bin, $i$, the probability to observe $n_{i}$ events is given by

\begin{equation}
 P_{i}(\mu) := P(d_{i}|\mu \cdot s_{i}+b_{i}) = \dfrac{(\mu \cdot s_{i}+b_{i})^{n_{i}} e^{-(\mu \cdot s_{i}+b_{i})} }{n_{i}!}
\label{eq:poisson-likelihood}
\end{equation}

where $s_{i}$ and $b_{i}$ are the total expected signal and background contributions. 
The likelihood for a search containing $N$ search regions is constructed as the product 
of the probabilities across the $N$ search regions, 

\begin{equation}
\mathcal{L}(\mu) = \prod_{i=1}^{N} P_{i}(\mu)
\label{eq:stat-likelihood}
\end{equation}

In most cases, the background contribution in each search region will not be known with perfect accuracy and is therefore 
subject to systematic uncertainties. These uncertainties are modelled by modifying the background contributions as 
$b_{i}\rightarrow b_{i}+\delta b_{i}$, where $\delta b_{i}$ are constrained nuisance parameters. The likelihood then takes the form

\begin{equation}
\mathcal{L}(\mu, \delta \mathbf{b}) = \prod_{i=1}^{N} P_{i}(\mu,\delta b_{i}) \cdot \mathrm{exp} \left[ \sum_{j=1}^{N}\sum_{k=1}^{N} (\delta b_{j}) V_{jk} (\delta b_{k}) \right]
\label{eq:full-likelihood}
\end{equation}

where $\delta\mathbf{b}=(\delta b_{1},\delta b_{2}...\delta b_{N})$ and $P_{i}(\mu,0)=P_{i}(\mu)$. The matrix element $V_{jk}$ represents the covariance between
the total expected background in the $j$--th and $k$--th search regions. The constraint 

It should be noted that the simplified likelihood presented in Equation~\ref{eq:full-likelihood} is an approximation to the full likelihood used 
by most CMS analyses in that it requires the following assumptions;

\begin{itemize}
\item{The constraints on the background contributions are Gaussian such that the distribution of the number of background events is symmetric about the expectation, $b_{i}$, 
and its variance is independent of $\delta \mathbf{b}$. Often, the background contributions are estimated from control regions in data with large sample sizes, which allows for this 
assumption to be made.}

\item{The linear correlation between the background contribution in each bin is sufficient to model the  pdf of the background expectations such that the constraint 
can be expressed as a multivariate Gaussian.}
\end{itemize}




\subsection{Procedure for deriving inputs}

To define the simplified likelihood the covariance and background predictions must be 
provided by the analysis. This procedure is identical if using the 
aggregated regions described in \ref{sec:aggregate-signal-region} or the nominal signal region. 
The simplified likelihood does not contain the control region 
section of the full likelihood. If the control regions are explicitly included in the analysis
the values and uncertainties of the nuisances ($\rho$ and $a_i$)
must be determined from a maximum likelihood fit considering only the control regions.
These values and uncertainties must then be used to derive the predictions and covariance
between the signal (or aggregate) region bins, $i$. If the control regions are not explicitly 
included then no fit is necessary. The predictions in each bin, $\mathbf{b_0}$, can be 
determined from the nuisances with values described above. To determine the covariance, pseudo-datasets are generated
by sampling the pdfs of the nuisances. The covariance may then be determined as in Equation~\ref{eq-cov}.

\begin{equation}
\sigma_{ij}=\sum^N_{t=1}{\frac{(b^t_i-b_{0,i})\times(b^t_j-b_{0,j})}{N}}
\label{eq-cov}
\end{equation}

where $\sigma_{ij}$ is the covariance between bins $i$ and $j$, $b^t_i$ is the
the pseudo-data in bin $i$i for pseudo-dataset t, 
$b_i$ is the prediction for bin $i$ and $N$ is the total number of generated pseudo-datasets.
Note that when the effect of neglecting the correlations is tested the off diagonal 
elements, $i\neqj$, are set to $\sigma_{ij} = 0$.


\subsection{Aggregating with covariance}

The covariance between bins provides an alternative method for aggregation. 
Using the results from the nominal signal regions the predictons and covariance of aggregate regions
may be determined as in Equation~\ref{eq-agg-cov}

\begin{align}
b_{0,I} = \sum_i b_{0,i} && \sigma_{IJ}=\sum_i\sum_j\sigma_{ij}
\label{eq-agg-cov}
\end{align}

where $b_{0,I}$ is the predicted background for the aggregate bin $I$,
$b_{0,i}$ is the predicted background in the original bin, $\sigma_{IJ}$
is the covariance between aggregate regions $I,J$ and $\sigma_{ij}$ is
the covariance in the original bins $i,j$. The sums are over all
bins being aggregated into aggregate regions $I,J$. The predictions
and covariance for the aggregate regions may then be used in defining 
the simplified likelihood in Equation~\ref{eq-simplified-likelihood}.
This method for aggregating is approximate as the uncertainties
on the bins being aggregated are treated as symmetric and as following
gaussian pdfs.



% covariance must be taken as the result of a maximum likelihood fit considering the control regions only. 
% If the control region is not included then the results fit may be taken. 
%%____________________________________________________________________________||
% \subsection{Motivation}
% \begin{itemize}
% \item Even with SSR recasters have insufficient information to reproduce analysis
% \item Full likelihood is overkill - what is necessary?
% \end{itemize}
% \subsection{Theory}
% \label{sec:sl-theory}
% \begin{itemize}
% \item Definition of simplified likelihood
% \item Inputs needed
% \item What's simplified? No signal systs, no CR, gaussian unc
% \end{itemize}
% \subsection{Procedure}
% \label{sec:sl-procedure}
% \begin{itemize}
% \item Determining correlation matrix
% \item Defining likelihood
% \item For study of impact setting off-diag elements to 0
% \end{itemize}
% \subsection{Signal contamination}
% \label{sec:signal-contamination}
% \begin{itemize}
% \item Definition of reduced efficiency method
% \item What do recasters need to take account of contamination?
% \end{itemize}
% \subsection{Results}
% \begin{itemize}
% \item Covariance matrix
% \item DeltaNLL vs r for example model (compare with + without correlations)
% \item Limit planes for several models (T1qqqq, T2tt, T2bb)
% \item Ratios + comparison to full
% \end{itemize}

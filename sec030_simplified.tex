%%____________________________________________________________________________||
\section{Simplified likelihood}
\label{sec:simplified-likelihood}

The results of the (signal region masked) fit with aggregate regions described
in Section~\ref{sec:aggregate-signal-regions} including predictions
and uncertainties can be given for recasters to interprete their own
models. However, many searches for new physics, including the \alphat,
analyses have complex correlations between bins. Corelations 
can be introduced through correlated systematic uncertainties as well as 
through signal bins which are predicted with the same (or systematically
correlated) control region bins. As will be shown, neglecting these 
correlations can lead to significant bias in limits. However,
the complexity of the true likelihood means it would be impractical
and time consuming for recasters for use in interpreting their own models.

This section details how analyses can release sufficient information for recasters to 
accurately reproduce the results of the true likelihood (with aggregate regions or
the full signal region) through the use of a simplified likelihood. This is defined
in Section~\ref{sec:sl-definition} and only requires the covariance matrix
in addition to the predictions from the analysis. The procedure for determining
this covariance and defining the simplified likelihood is outlined in Section~\ref{sec:sl-procedure}.
Section~\ref{sec:red-eff} describes how signal contamination may be accounted for
through the use of the reduced efficiency method \cite{redeff}. Finally, results
from the \alphat analysis are shown in Section~\ref{sec:results} including comparisons
to the results of neglecting the correlations.


\subsection{Definition of the simplified likelihood}
\label{sec:sl-definition}
The true likelihood for the full signal region is defined in Equations~\ref{eq:hadronicLikelihood}
and \label{eq:controlLikelihood} and for the aggregate regions the 
hadronic section is redefined to Equation~\ref{eq:agg-hadronicLikelihood}. The uncertainty
of each signal region bin, $i$, as well as its correlation to all other signal region bins, $j$, 
due to both systematic uncertainties and connection to the control region may be represented
by elements of the symmetric covariance matrix $\mathbf{V}_{i,j}$. Where diagonal elements encode the 
variance while off-diagonal elements encode the correlation between bins. The simplified likelihood
may then be written as in Equation~\ref{eq:simplified-likelihood}.

\begin{equation}
\mathcal{L}_{\mathrm{simplified}}=\prod_i(\mathrm{Pois}(n_{i} |\, b_{i} + s_{i}\times\mu)
)\times\exp((\mathbf{b}-\mathbf{b}_{0})^T\mathbf{V}^{-1}(\mathbf{b}-\mathbf{b}_{0}))
\label{eq:simplified-likelihood}
\end{equation}

Where signal systematics have been dropped, $\mathbf{b}_{\mathrm{signal}}$ is the vector
of background predictions in each bin, $\mathbf{b}_{\mathrm{signal},0}$ are their central values 
and $\mathbf{V}$ is the covariance matrix. The procedure for deriving the predictions and covariance
is described in \ref{sec:sl-procedure}. The likelihood is then minimized with respect to $\mu$ and $\mathbf{b}$.

The simplified likelihood is an approximation as the nuisances, including the connections
to the control regions are treated as symmetric and following gaussian pdfs.
Nuisances which are not well described by the gaussian approximation, for example 
to asymmetries or significantly different behaviour in the tails, may not be so well modelled.
Signal systematics are also neglected, although these could be included as a simple
extension if desired. Despite the loss of the details of the systematic modelling 
as well as the explicit links to the control regions this simplified likelihood will
be shown to be a good approximation of the true likelihood for the example of the
\alphat analysis. 

\subsection{Procedure for deriving inputs}

To define the simplified likelihood the covariance and background predictions must be 
provided by the analysis. This procedure is identical if using the 
aggregated regions described in \ref{sec:aggregate-signal-region} or the nominal signal region. 
The simplified likelihood does not contain the control region 
section of the full likelihood. If the control regions are explicitly included in the analysis
the values and uncertainties of the nuisances ($\rho$ and $a_i$)
must be determined from a maximum likelihood fit considering only the control regions.
These values and uncertainties must then be used to derive the predictions and covariance
between the signal (or aggregate) region bins, $i$. If the control regions are not explicitly 
included then no fit is necessary. The predictions in each bin, $\mathbf{b_0}$, can be 
determined from the nuisances with values described above. To determine the covariance, pseudo-datasets are generated
by sampling the pdfs of the nuisances. The covariance may then be determined as in Equation~\ref{eq-cov}.

\begin{equation}
\sigma_{ij}=\sum^N_{t=1}{\frac{(b^t_i-b_{0,i})\times(b^t_j-b_{0,j})}{N}}
\label{eq-cov}
\end{equation}

where $\sigma_{ij}$ is the covariance between bins $i$ and $j$, $b^t_i$ is the
the pseudo-data in bin $i$i for pseudo-dataset t, 
$b_i$ is the prediction for bin $i$ and $N$ is the total number of generated pseudo-datasets.
Note that when the effect of neglecting the correlations is tested the off diagonal 
elements, $i\neqj$, are set to $\sigma_{ij} = 0$.


\subsection{Aggregating with covariance}

The covariance between bins provides an alternative method for aggregation. 
Using the results from the nominal signal regions the predictons and covariance of aggregate regions
may be determined as in Equation~\ref{eq-agg-cov}

\begin{align}
b_{0,I} = \sum_i b_{0,i} && \sigma_{IJ}=\sum_i\sum_j\sigma_{ij}
\label{eq-agg-cov}
\end{align}

where $b_{0,I}$ is the predicted background for the aggregate bin $I$,
$b_{0,i}$ is the predicted background in the original bin, $\sigma_{IJ}$
is the covariance between aggregate regions $I,J$ and $\sigma_{ij}$ is
the covariance in the original bins $i,j$. The sums are over all
bins being aggregated into aggregate regions $I,J$. The predictions
and covariance for the aggregate regions may then be used in defining 
the simplified likelihood in Equation~\ref{eq-simplified-likelihood}.
This method for aggregating is approximate as the uncertainties
on the bins being aggregated are treated as symmetric and as following
gaussian pdfs.



% covariance must be taken as the result of a maximum likelihood fit considering the control regions only. 
% If the control region is not included then the results fit may be taken. 
%%____________________________________________________________________________||
% \subsection{Motivation}
% \begin{itemize}
% \item Even with SSR recasters have insufficient information to reproduce analysis
% \item Full likelihood is overkill - what is necessary?
% \end{itemize}
% \subsection{Theory}
% \label{sec:sl-theory}
% \begin{itemize}
% \item Definition of simplified likelihood
% \item Inputs needed
% \item What's simplified? No signal systs, no CR, gaussian unc
% \end{itemize}
% \subsection{Procedure}
% \label{sec:sl-procedure}
% \begin{itemize}
% \item Determining correlation matrix
% \item Defining likelihood
% \item For study of impact setting off-diag elements to 0
% \end{itemize}
% \subsection{Signal contamination}
% \label{sec:signal-contamination}
% \begin{itemize}
% \item Definition of reduced efficiency method
% \item What do recasters need to take account of contamination?
% \end{itemize}
% \subsection{Results}
% \begin{itemize}
% \item Covariance matrix
% \item DeltaNLL vs r for example model (compare with + without correlations)
% \item Limit planes for several models (T1qqqq, T2tt, T2bb)
% \item Ratios + comparison to full
% \end{itemize}

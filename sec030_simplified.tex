%%____________________________________________________________________________||
\section{Simplified likelihood}
\label{sec:simplified-likelihood}

This section describes a procedure for using the information provided by the CMS Collaboration to 
re-interpret searches for new physics through the use of a simplified likelihood. The likelihood is 
constructed from a product of counting experiments, representing each search region in one or more search regions. 
For a given region, $i$, the probability to observe $n_{i}$ events is given by

\begin{equation}
 P_{i}(\mu) := P(d_{i}|\mu \cdot s_{i}+b_{i}) = \dfrac{(\mu \cdot s_{i}+b_{i})^{n_{i}} e^{-(\mu \cdot s_{i}+b_{i})} }{n_{i}!}
\label{eq:poisson-likelihood}
\end{equation}

where $s_{i}$ and $b_{i}$ are the total expected signal and background contributions. 
The likelihood for a search containing $N$ search regions is constructed as the product 
of the probabilities across the $N$ search regions, 

\begin{equation}
\mathcal{L}(\mu) = \prod_{i=1}^{N} P_{i}(\mu)
\label{eq:stat-likelihood}
\end{equation}

In most cases, the background contribution in each search region will not be known with perfect accuracy and is therefore 
subject to systematic uncertainties. These uncertainties are modelled by modifying the background contributions as 
$b_{i}\rightarrow b_{i}+\delta b_{i}$, where $\delta\mathbf{b}=(\delta b_{1},\delta b_{2}...\delta b_{N})$ are constrained nuisance parameters. The likelihood then takes the form

\begin{equation}
\mathcal{L}(\mu, \delta \mathbf{b}) = \prod_{i=1}^{N} P_{i}(\mu,\delta b_{i}) \cdot \mathrm{exp} \left[ \sum_{j=1}^{N}\sum_{k=1}^{N} (\delta b_{j}) V_{jk} (\delta b_{k}) \right]
\label{eq:full-likelihood}
\end{equation}

where $P_{i}(\mu,0)=P_{i}(\mu)$ and $V_{jk}$ represents the covariance between
the total expected background in the $j$--th and $k$--th search regions. It should be noted that 
the simplified likelihood presented in Equation~\ref{eq:full-likelihood} is an approximation to the full likelihood used 
by most CMS analyses in that it requires the following assumptions;

\begin{itemize}
\item{The constraints on the background contributions are Gaussian such that the distribution of the number of background events is symmetric about the expectation, $b_{i}$, 
and its variance is independent of $\delta \mathbf{b}$. Often, the background contributions are estimated from control regions in data with large sample sizes, which allows for this 
assumption to be made.}

\item{The linear correlation between the background contribution in each region is sufficient to model the  pdf of the background expectations such that the constraint 
can be expressed as a multivariate Gaussian.}
\end{itemize}


From the above description, it should be clear that there are three ingredients which are 
required in order to construct the simplified likelihood. These ingredients are as follows; 

\begin{itemize}
\item {the data count in each search region}
\item {the background and signal expectations, former from the experiment, latter from Delphes}
\item {the covariance between search regions for the background. Often derived from control regions.}
\end{itemize}



% covariance must be taken as the result of a maximum likelihood fit considering the control regions only. 
% If the control region is not included then the results fit may be taken. 
%%____________________________________________________________________________||
% \subsection{Motivation}
% \begin{itemize}
% \item Even with SSR recasters have insufficient information to reproduce analysis
% \item Full likelihood is overkill - what is necessary?
% \end{itemize}
% \subsection{Theory}
% \label{sec:sl-theory}
% \begin{itemize}
% \item Definition of simplified likelihood
% \item Inputs needed
% \item What's simplified? No signal systs, no CR, gaussian unc
% \end{itemize}
% \subsection{Procedure}
% \label{sec:sl-procedure}
% \begin{itemize}
% \item Determining correlation matrix
% \item Defining likelihood
% \item For study of impact setting off-diag elements to 0
% \end{itemize}
% \subsection{Signal contamination}
% \label{sec:signal-contamination}
% \begin{itemize}
% \item Definition of reduced efficiency method
% \item What do recasters need to take account of contamination?
% \end{itemize}
% \subsection{Results}
% \begin{itemize}
% \item Covariance matrix
% \item DeltaNLL vs r for example model (compare with + without correlations)
% \item Limit planes for several models (T1qqqq, T2tt, T2bb)
% \item Ratios + comparison to full
% \end{itemize}

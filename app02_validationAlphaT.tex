%%____________________________________________________________________________||
\section{Validation with the \alphat analysis}
The \alphat analysis is an inclusive search for supersymmetry which
uses a fine categorisation in \nj,\nb,\scalht,and \mht to provide
sensitivity to a wide range of new physics models. The results 
of this search for 12.9\fb of data at 13TeV from Run 2 of the LHC
are described fully in \cite{CMS-PAS-SUS-16-016}. In this section,
the use of aggregate regions is discribed for the \alphat analysis, as well as a 
validation of the simplified likelihood method.

\subsection{Application of the the aggregate regions with the \alphat analysis}

The final \nj,\nb categories and \scalht binning used for the full likelihood 
are summarised in Table~\ref{tab:binning}. These bins are further split into
$100 \GeV$ \mht bins as defined in \cite{alphaT}. The control region
predictions are inclusive in \mht but binned according to Table~\ref{tab:binning}.

\begin{table}[htb!]
  \topcaption{Summary of the lower bounds of the first and final bins
    in \scalht (the latter in parentheses) as a function of \njet and
    \nb. Intermediate \scalht bins are taken from 200,250,300,400,500,600,700,$>800\GeV$} 
  \label{tab:binning}
  \centering
  \footnotesize
  \begin{tabular}{ lrrrr }
    \hline
%    \njet                   & \multicolumn{4}{c}{\nb}                                           \\
%    \cline{2-5}
%                            & 0         & 1         & 2         & $\geq$3                       \\
    $\njet \backslash\, \nb$ & 0         & 1         & 2         & $\geq$3                       \\
    \hline
    \multicolumn{5}{l}{\bf Monojet}                                                              \\
    1                        & 200 (600) & 200 (500) & -     & -                         \\
    \multicolumn{5}{l}{\bf Asymmetric}                                                           \\
    2                        & 200 (600) & 200 (500) & 200 (400) & -                         \\
    3                        & 200 (600) & 200 (600) & 200 (500) & 200 (300)                     \\
    4                        & 200 (600) & 200 (600) & 200 (600) & 250 (400)                     \\
    $\geq$5                  & 250 (600) & 250 (600) & 250 (600) & 300 (500)                     \\
    \multicolumn{5}{l}{\bf Symmetric}                                                            \\
    2                        & 200 (800) & 200 (800) & 200 (600) & -                         \\
    3                        & 200 (800) & 250 (800) & 250 (800) & \phantom{0}-\phantom{0} (250) \\
    4                        & 300 (800) & 300 (800) & 300 (800) & 300 (800)                     \\
    $\geq$5                  & 350 (800) & 350 (800) & 350 (800) & 350 (800)                     \\
    \hline
  \end{tabular}
\end{table}

The $\mathcal{O}800$ bins provide generic sensitivity but are not optimal
for simple reinterpretation. A simplification can be made that maintains
sensitivity to a large class of models through the use of aggregate regions.
The final categories are shown in Table~\ref{tab:agg-binnning}.
These are defined to be disjoint, contigous and to cover the full
signal region and so in combination to reflect sensitivity from 
full signal region phase space. In addition
for each aggregate category eight exclusive \mht bins with
lower bounds $100,200,300,400,500,600,700,\ge800GeV$ are defined.

\begin{table}[tb]
  \topcaption{Aggregate region bins. The \scalht dimension is
  merged to $\geq200\GeV$, \nb to two categories of \nb = $0,1$ and $\geq2$. 
  The merged \nj~ categories are summarised in this table. Each category is
  further binned using eight \mht bins with lower bounds from $100-800\GeV$.}
  \label{tab:agg-binning}
  \centering
  \footnotesize
  \begin{tabular}{ llll }
    \hline
    \nj topology & \multicolumn{3}{l}{Merged jet categories} \\
    \hline
     & \bf Monojet & \bf Asymmetric& \bf Symmetric \\
    Monojet-like & 1 & 2 & 2                         \\
    Asymmetric high \nj& - & 3, 4, $\geq5$ & -                 \\
    Mid \nj & - &  3, 4 & -                        \\
    High \nj & - & $\geq5$ & -                     \\
    \hline
  \end{tabular}
\end{table}

The jet categories are merged to four seperate categories motivated by their sensitivity to 
different new physics topologies. For example, the Monojet-like topology is targeted towards
dark matter models and compressed spectra while the high \nj topology targets 
uncompressed gluino and squark models.
The b-jet categories are combined as $\nb=0,1$ \nb models and $\nb\geq2$ targeted at
light and heavy flavour new physics respectively. Finally, the \scalht dimension is 
entirely merged as the \mht dimension generally provides better sensitivity
for new physics models. The use of the
aggregate regions ensures that the components of the background are still
corrected based on the nominal, finely binned regions with relevant systematics.

\subsection{Results using aggregate regions for the \alphat analysis}

The aggregate regions provide an easily comprehensible
overview of the signal region when compared to the results
of the nominal signal bins. Figure~\ref{fig:aggFitResult} shows
the post fit predictions in the mid and high njet categories
for the aggregate regions compared to the observed data.

To evaluate the effect on the reach of the \alphat analysis the expected and observed 95\% upper limit
on the signal strength, defined as in \cite{limit-stuff}, for the full signal region 
is compared that using the aggregate regions. The limits are shown side by side
in Fig~\ref{fig:limit-planes} for three models, T2tt T2bb and T1bbbb. Where the mass
splittings are small the expected limits typically reduce by $\mathcal{O} 100\GeV$ 
compared to the full signal region for all models  

The predictions from the aggregate regions
can be used to provide predictions and uncertainties
to allow recasters to interpret their own models. In the case of the \alphat search
this is significantly easier than recasting the full signal region. To make best use of this information, 
as will be discussed in Section~\ref{sec:simplified-likelihood}, 
the covariance matrix encoding the level of correlation between bins must also be utilised.
\clearpage
\begin{figure}[!tbhp]
    \caption{ Signal region predictions and data observations for the aggregate regions. 
    The predictions are made using a fit to the control region only. \label{fig:aggFitResult} }
  \begin{center}
    \subfigure[Mid \nj, $\nb \leq 1$]   { \includegraphics[width=0.4\textwidth]{figures/alphaT/agg_fitResults/mhtShape_le1b_ge3j_200_Inf_crfit_aux.pdf} } ~~
    \subfigure[Mid \nj, $\nb \geq 2$]{ \includegraphics[width=0.4\textwidth]{figures/alphaT/agg_fitResults/mhtShape_ge2b_ge3j_200_Inf_crfit_aux.pdf} } \\
    \subfigure[High \nj, $\nb \leq 1$]   { \includegraphics[width=0.4\textwidth]{figures/alphaT/agg_fitResults/mhtShape_le1b_ge5j_200_Inf_crfit_aux.pdf} } ~~
    \subfigure[High \nj, $\nb \geq 2$]{ \includegraphics[width=0.4\textwidth]{figures/alphaT/agg_fitResults/mhtShape_ge2b_ge5j_200_Inf_crfit_aux.pdf} } \\
  \end{center}
\end{figure}


\clearpage
\begin{figure}[!tbhp]
    \caption{ Limit planes shown for both the full signal regions (left) and the aggregate regions (right).\label{fig:limit-planes}. 
    For the compressed models. }
  \begin{center}
    \subfigure[T2bb full signal region]{ \includegraphics[width=0.45\textwidth]{figures/alphaT/limitPlanesNominal/SUS16T2bbXSEC} } ~~
    \subfigure[T2bb aggregate regions]{ \includegraphics[width=0.45\textwidth]{figures/alphaT/limitPlanesAgg/SUS16T2bbXSEC} } \\
    \subfigure[T2tt full signal region]   { \includegraphics[width=0.45\textwidth]{figures/alphaT/limitPlanesNominal/SUS16T2ttXSEC} } ~~
    \subfigure[T2tt aggregate regions]{ \includegraphics[width=0.45\textwidth]{figures/alphaT/limitPlanesAgg/SUS16T2ttXSEC} } \\
    \subfigure[T1bbbb full signal region]   { \includegraphics[width=0.45\textwidth]{figures/alphaT/limitPlanesNominal/SUS16T1bbbbXSEC} } ~~
    \subfigure[T1bbbb aggregate regions]{ \includegraphics[width=0.45\textwidth]{figures/alphaT/limitPlanesAgg/SUS16T1bbbbXSEC} } \\
  \end{center}
\end{figure}

\subsection{Validation of the simplified likelihood}

The covariance and predictions of the aggregate regions for the \alphat analysis may then 
be used to validate the simplified likelihood. Figure~\ref{fig:likelihoodscan-alphaT} shows the value of $q(\mu)$ as a function of $\mu$ for 
a benchmark model which has substantial contribution in many different signal regions 
(T2tt mStop = , mLSP = ).The values when $q(\mu)$ is defined using the likelihood of Equation~\ref{eq:full-likelihood} 
are shown and compared with the same definition but assuming no correlations between the 
background yields by setting $V_{ij}=0$ for $i\neq j$. The results using the full likelihood (with aggregate regions and no signal systematics) 
used by the \alphat analysis are also shown. The simplified likelihood shows good agreement with the full likelihood 
while it is clear that ignoring the correlations results in a bias of the estimate of $\hat{\mu}$. 

\begin{figure}[hbt]
  \begin{center} 
   \includegraphics[width=1.5\cmsFigWidth]{figures/alphaT/rAT.pdf}
   \caption{The value of $q(\mu)$ for the \alphaT analysis defined using the simplified likelihood using the full covariance matrix (open blue points), assuming no correlations between the 
   background yields (open magenta crosses) and defined using the full likelihood (solid black line).}
   \label{fig:likelihoodscanAT} 
  \end{center}
\end{figure}

In Figure~\ref{fig:limitPlanes} the ratio between the limit with the simplified and the full likelihood is shown.
The contours of $\mu=1$ excluded at $95\%$ for the full likelihood, the simplified likelihood
and the simplified likelihood where correlations are neglected are overlaid. When correlations
are considered the simplified and full likelihood provide comparable results, however, if 
these are neglected a significant bias is observed.

\begin{figure}[hbt]
  \begin{center} 
   \includegraphics[width=1.5\cmsFigWidth]{figures/alphaT/full_T2bb_obs}
   \caption{Ratio of the $95\%$ upper limits for the simplified likelihood/full likelihood.
   The contour of $\mu=1$ for the simplified likelihood using the full covariance matrix (solid red line), 
   the simplified likelihood assuming no correlations between the background yields (solid black line) and the full
   likelihood (dotted black line) are overlaid.
   }
   \label{fig:limitPlanes} 
  \end{center}
\end{figure}
%%____________________________________________________________________________||






%%____________________________________________________________________________||


\subsection{Procedure for deriving inputs}

To define the simplified likelihood the covariance and background predictions must be 
provided by the analysis. This procedure is identical if using the 
aggregated regions described in \ref{sec:aggregate-signal-region} or the nominal signal region. 
The simplified likelihood does not contain the control region 
section of the full likelihood. If the control regions are explicitly included in the analysis
the values and uncertainties of the nuisances ($\rho$ and $a_i$)
must be determined from a maximum likelihood fit considering only the control regions.
These values and uncertainties must then be used to derive the predictions and covariance
between the signal (or aggregate) region bins, $i$. If the control regions are not explicitly 
included then no fit is necessary. The predictions in each bin, $\mathbf{b_0}$, can be 
determined from the nuisances with values described above. To determine the covariance, pseudo-datasets are generated
by sampling the pdfs of the nuisances. The covariance may then be determined as in Equation~\ref{eq-cov}.

\begin{equation}
\sigma_{ij}=\sum^N_{t=1}{\frac{(b^t_i-b_{0,i})\times(b^t_j-b_{0,j})}{N}}
\label{eq-cov}
\end{equation}

where $\sigma_{ij}$ is the covariance between bins $i$ and $j$, $b^t_i$ is the
the pseudo-data in bin $i$i for pseudo-dataset t, 
$b_i$ is the prediction for bin $i$ and $N$ is the total number of generated pseudo-datasets.
Note that when the effect of neglecting the correlations is tested the off diagonal 
elements, $i\neqj$, are set to $\sigma_{ij} = 0$.


\subsection{Aggregating with covariance}

The covariance between bins provides an alternative method for aggregation. 
Using the results from the nominal signal regions the predictons and covariance of aggregate regions
may be determined as in Equation~\ref{eq-agg-cov}

\begin{align}
b_{0,I} = \sum_i b_{0,i} && \sigma_{IJ}=\sum_i\sum_j\sigma_{ij}
\label{eq-agg-cov}
\end{align}

where $b_{0,I}$ is the predicted background for the aggregate bin $I$,
$b_{0,i}$ is the predicted background in the original bin, $\sigma_{IJ}$
is the covariance between aggregate regions $I,J$ and $\sigma_{ij}$ is
the covariance in the original bins $i,j$. The sums are over all
bins being aggregated into aggregate regions $I,J$. The predictions
and covariance for the aggregate regions may then be used in defining 
the simplified likelihood in Equation~\ref{eq-simplified-likelihood}.
This method for aggregating is approximate as the uncertainties
on the bins being aggregated are treated as symmetric and as following
gaussian pdfs.
